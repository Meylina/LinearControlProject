All time derivatives are set to zero in order to determine the stationary states. Therefore, we have $\frac{dx}{dt} = 0$ and equations \eqref{eq:1.1} and \eqref{eq:1.2} are rewritten below:
\begin{align}
   u_{e} &= R_{e}i \label{stateq1} \\
   Bl(x)i &= k(x)x \label{stateq2}
\end{align}

As $u_{e} = 0$, from \eqref{stateq1} we obtain $i = 0$ and we deduce by substituting in \eqref{stateq2} that $k(x)x = 0$. Then $k(x) = 0$ or $x = 0$. The discriminant of the polynomial $k(x)$ of degree 2 is $\Delta = k_{1}^2-4k_{2}k_{0} < 0$. The voice coil displacement $x$ being real, we discard this value and get:
\begin{equation*}
\text{x}_{0}=
\begin{pmatrix}
  0 \\
  0 \\
  0
\end{pmatrix}
\end{equation*}

We linearise the model $\dot{\text{x}} = h(\text{x},\text{u})$ with $h(\text{x}) = f(\text{x}) + g(\text{x})\text{u}$ around the stationary states:
\begin{align*}
   x(t) &= x_{0} + \Delta x(t) = \Delta x(t)\\
   \dot{x}(t) &= \dot{x}_{0} + \Delta \dot{x}(t) = \Delta \dot{x}(t) \\
   i(t) &= i_{0} + \Delta i(t) = \Delta i(t)
\end{align*}

The linear model can then be written in the form:
\begin{align*}
   \dot{\text{x}} &= A\text{x} + B\text{u} \\
   y &= C\text{x}
\end{align*}

where \begin{equation*} A = \begin{pmatrix}
  \frac{\partial{h_{1}}}{\partial{\text{x}_{1}}} & \frac{\partial{h_{1}}}{\partial{\text{x}_{2}}} & \frac{\partial{h_{1}}}{\partial{\text{x}_{3}}} \\
  \frac{\partial{h_{2}}}{\partial{\text{x}_{1}}} & \frac{\partial{h_{2}}}{\partial{\text{x}_{2}}} & \frac{\partial{h_{2}}}{\partial{\text{x}_{3}}} \\
  \frac{\partial{h_{3}}}{\partial{\text{x}_{1}}} & \frac{\partial{h_{3}}}{\partial{\text{x}_{2}}} & \frac{\partial{h_{3}}}{\partial{\text{x}_{3}}}
\end{pmatrix}_{\text{x}_{0}}
\qquad B = \begin{pmatrix}
\frac{\partial{h_{1}}}{\partial{\text{u}}} \\
\frac{\partial{h_{2}}}{\partial{\text{u}}} \\
\frac{\partial{h_{3}}}{\partial{\text{u}}}
\end{pmatrix}_{\text{x}_{0}}
\qquad C = \begin{pmatrix}
\frac{\partial{r}}{\partial{\text{x}_{1}}} & \frac{\partial{r}}{\partial{\text{x}_{2}}} & \frac{\partial{r}}{\partial{\text{x}_{3}}} 
\end{pmatrix}_{\text{x}_{0}}
\end{equation*}

We obtain:
\begin{equation*} 
\frac{\partial{h_{1}}}{\partial{\text{x}_{1}}} = 0 \qquad \frac{\partial{h_{1}}}{\partial{\text{x}_{2}}} = 1 \qquad \frac{\partial{h_{1}}}{\partial{\text{x}_{3}}} = 0
\end{equation*}
\begin{equation*}
\frac{\partial{h_{2}}}{\partial{\text{x}_{1}}} = \frac{b_{1}\text{x}_{30} + 2b_{2}\text{x}_{10}\text{x}_{30} - (k_{0}+2k_{1}\text{x}_{10}+3k_{2}\text{x}_{10}^2) + l_{2}\text{x}_{20}\text{x}_{30}^2}{m_{t}}
\end{equation*}
\begin{equation*}
\frac{\partial{h_{2}}}{\partial{\text{x}_{2}}} = \frac{-R_{m} + \frac{1}{2}\times (l_{1}+2l_{2}\text{x}_{10})\text{x}_{30}^2}{m_{t}}
\end{equation*}
\begin{equation*}
\frac{\partial{h_{2}}}{\partial{\text{x}_{3}}} = \frac{Bl_{0}+b_{1}\text{x}_{10}+b_{2}\text{x}_{10}^2+(l_{1}+2l_{2}\text{x}_{10})\text{x}_{20}\text{x}_{30}}{m_{t}}
\end{equation*}
\begin{multline*}
\frac{\partial{h_{3}}}{\partial{\text{x}_{1}}} = \frac{(-2l_{2}\text{x}_{20}^2\text{x}_{30}-(b_{1}+2b_{2}\text{x}_{10})\text{x}_{20}) \times (L_{e0}+l_1\text{x}_{10}+l_2\text{x}_{10}^2)}{(L_{e0}+l_1\text{x}_{10}+l_2\text{x}_{10}^2)^2} - (l_{1}+2l_{2}\text{x}_{10}) \times \\
\frac{(-(R_e+(l_1+2l_2\text{x}_{10})\text{x}_{20}^2)\text{x}_{30}-(Bl_0+b_1\text{x}_{10}+b_2\text{x}_{10}^2)\text{x}_{20})}{(L_{e0}+l_1\text{x}_{10}+l_2\text{x}_{10}^2)^2}
\end{multline*}
\begin{equation*}
\frac{\partial{h_{3}}}{\partial{\text{x}_{2}}} = - \frac{2(l_1+2l_2\text{x}_{10})\text{x}_{20}\text{x}_{30}+Bl_0+b_1\text{x}_{10}+b_2\text{x}_{10}^2}{L_{e0}+l_1\text{x}_{10}+l_2\text{x}_{10}^2}
\end{equation*}
\begin{equation*}
\frac{\partial{h_{3}}}{\partial{\text{x}_{3}}} = - \frac{(l_1+2l_2\text{x}_{10})\text{x}_{20}^2+R_{e}}{L_{e0}+l_1\text{x}_{10}+l_2\text{x}_{10}^2}
\end{equation*}
\begin{equation*}
\frac{\partial{h_{1}}}{\partial{\text{u}}} = 0 \qquad 
\frac{\partial{h_{2}}}{\partial{\text{u}}} = 0 \qquad
\frac{\partial{h_{3}}}{\partial{\text{u}}} = \frac{1}{L_{e0}+l_1\text{x}_{10}+l_2\text{x}_{10}^2}
\end{equation*}
\begin{equation*}
\frac{\partial{r}}{\partial{\text{x}_{1}}} = 0 \qquad 
\frac{\partial{r}}{\partial{\text{x}_{2}}} = 0 \qquad
\frac{\partial{r}}{\partial{\text{x}_{3}}} = 1
\end{equation*}\\
We then substitute $\text{x}_{10} = \text{x}_{20} = \text{x}_{30} = 0$.
Finally, we get:
\begin{equation*} A = \begin{pmatrix}
   0 & 1 & 0 \\
   -\frac{k_0}{m_t} & -\frac{R_m}{m_t} & \frac{Bl_0}{m_t} \\
  0 & - \frac{Bl_0}{L_{e0}} & - \frac{R_e}{L_{e0}}  
\end{pmatrix}
\qquad B = \begin{pmatrix}
0 \\
0 \\
\frac{1}{L_{e0}}
\end{pmatrix}
\qquad C = \begin{pmatrix}
	0 & 0 & 1 
\end{pmatrix}
\end{equation*}\\
Numerically
\begin{equation*} A = \begin{pmatrix}
   0 & 1 & 0 \\
  -1.20 \cdot 10^5 & -50.46 & 279.19 \\
  0 & -2.57 \cdot 10^3 & -1.40 \cdot 10^3
\end{pmatrix}
\qquad B = \begin{pmatrix}
0 \\
0 \\
177
\end{pmatrix}
\qquad C = \begin{pmatrix}
	0 & 0 & 1 
\end{pmatrix}
\end{equation*}\\
We can check these results with the matlab function \textit{linmod(model,$\text{x}_0$,$u_e$)}:
\vspace*{-0.5cm}
\begin{lstlisting}[language=Matlab]
[A,B,C,D] = linmod('nonLinearModel',[0;0;0],0);
\end{lstlisting}