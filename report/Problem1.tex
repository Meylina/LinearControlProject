By means of Eqs~\ref{eq:1.1},~\ref{eq:1.2},~\ref{eq:1.3},~\ref{eq:1.4} and~\ref{eq:1.5}, we can identify 3 state variables $x$,  $\dot{x}$ and $i$. We can also identify the input $u_e$.

$\text{x}=\begin{pmatrix}
   x \\
   \dot{x} \\
	 i
\end{pmatrix}$ and $\text{u}=(u_e)$

Then, we can derive the nonlinear dynamical state space model to obtain

\begin{align}
   \dot{x} &= \dot{x}\\
	 \ddot{x} &= \frac{(Bl_0+b_1x+b_2x^2)i-R_m\dot{x}-(k_0+k_1x+k_2x^2)x+\frac{1}{2}(l_1+2l_2x)\dot{x}i^2}{m_t}\\
	 \dot{i} &= \frac{u_e-(R_e+(l_1+2l_2x)\dot{x}^2)i-(Bl_0+b_1x+b_2x^2)\dot{x}}{L_{e0}+l_1x+l_2x^2}
\end{align}

In matrix format, we have

\begin{equation}
	\label{eq:eqModel}
	\dot{\text{x}}=f(\text{x})+g(\text{x})\text{u}
\end{equation}
with
\begin{equation}
	\label{eq:f(x)}
	f(\text{x})=\begin{pmatrix}
   \text{x}(2) \\
	 \frac{(Bl_0+b_1\text{x}(1)+b_2\text{x}(1)^2)\text{x}(3)-R_m\text{x}(2)-(k_0+k_1\text{x}(1)+k_2\text{x}(1)^2)\text{x}(1)+\frac{1}{2}(l_1+2l_2\text{x}(1))\text{x}(2)\text{x}(3)^2}{m_t}\\
	 \frac{-(R_e+(l_1+2l_2\text{x}(1))\text{x}(2)^2)\text{x}(3)-(Bl_0+b_1\text{x}(1)+b_2\text{x}(1)^2)\text{x}(2)}{L_{e0}+l_1\text{x}(1)+l_2\text{x}(1)^2}
\end{pmatrix}
\end{equation}
\begin{equation}
	\label{eq:g(u)}
	g(\text{x})=\begin{pmatrix}
   0\\
   0 \\
	 1
\end{pmatrix}
\end{equation}
