We can derive analytically the eigenvalues of the system dynamical matrix $A$.

\begin{align}
	\label{eq:eigenValues}
	det(\lambda I-A) & =det\begin{pmatrix}
 \lambda & -1 & 0 \\
  \frac{k_0}{m_t} & \lambda+\frac{R_m}{m_t} & -\frac{Bl_0}{m_t} \\
  0 & \frac{Bl_0}{Le_0} & \lambda+\frac{Re}{Le_0}
\end{pmatrix} \\
& = \lambda^3 +\left(\frac{Re}{Le_0}+\frac{Rm}{m_t}\right)\lambda^2+\left(\frac{RmRe+Bl_0^2}{m_tLe_0}+\frac{k_0}{m_t}\right)\lambda+\frac{k_0Re}{m_tLe_0}
\end{align}

We therefore have a third-order polynomial which can be solved with MATLAB but the eigenvalues can also be calculated using
\begin{lstlisting}[language=Matlab]
lambda = eig(A);
\end{lstlisting}

Thus, we obtain
\begin{equation}
	\label{eq:eigenValuesValues}
	\lambda = 1,0.10^2\begin{pmatrix}
	-2.9573 \\
	-5.7648 + 4.8571i\\
	-5.7648 - 4.8571i
	\end{pmatrix}
\end{equation}

We can notice that
\begin{equation}
	\label{eq:eigenValuesRe}
	Re(\lambda_i)<0, \ i=1,2,3
\end{equation}

which means that the system is asymptotically stable. Moreover, using Euler decomposition, the eigenmodes of the system are $e^{\lambda_1t}$, $e^{\Re(\lambda_2)t}\left(\cos(\Im(\lambda_2)t)+\sin(\Im(\lambda_2)t)\right)$ and $e^{\Re(\lambda_3)t}\left(\cos(\Im(\lambda_3)t)-\sin(\Im(\lambda_3)t)\right)$. We can notice that \begin{equation*}
\frac{1}{\lambda_1}=\tau_1>\tau_2, \tau_3, \qquad with\ \tau_i = \frac{1}{\sqrt{\Re(\lambda_i)^2+\Im(\lambda_i)^2}}, \qquad i = 2, 3
\end{equation*}
which means that the response of the state $x$ to an input $u_e$ will be slower than the one of $i$ and $\dot{x}$. We can also see that as $\lambda_2$ and $\lambda_3$ have an imaginary part which means that the eigenmodes associated to $\lambda_2$ and $\lambda_3$ will have an oscillatory behavior.
%$\dot{x}$ and $i$ will have an oscillatory response.
